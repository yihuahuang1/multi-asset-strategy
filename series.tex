% Options for packages loaded elsewhere
\PassOptionsToPackage{unicode}{hyperref}
\PassOptionsToPackage{hyphens}{url}
%
\documentclass[
]{book}
\usepackage{amsmath,amssymb}
\usepackage{iftex}
\ifPDFTeX
  \usepackage[T1]{fontenc}
  \usepackage[utf8]{inputenc}
  \usepackage{textcomp} % provide euro and other symbols
\else % if luatex or xetex
  \usepackage{unicode-math} % this also loads fontspec
  \defaultfontfeatures{Scale=MatchLowercase}
  \defaultfontfeatures[\rmfamily]{Ligatures=TeX,Scale=1}
\fi
\usepackage{lmodern}
\ifPDFTeX\else
  % xetex/luatex font selection
\fi
% Use upquote if available, for straight quotes in verbatim environments
\IfFileExists{upquote.sty}{\usepackage{upquote}}{}
\IfFileExists{microtype.sty}{% use microtype if available
  \usepackage[]{microtype}
  \UseMicrotypeSet[protrusion]{basicmath} % disable protrusion for tt fonts
}{}
\makeatletter
\@ifundefined{KOMAClassName}{% if non-KOMA class
  \IfFileExists{parskip.sty}{%
    \usepackage{parskip}
  }{% else
    \setlength{\parindent}{0pt}
    \setlength{\parskip}{6pt plus 2pt minus 1pt}}
}{% if KOMA class
  \KOMAoptions{parskip=half}}
\makeatother
\usepackage{xcolor}
\usepackage{color}
\usepackage{fancyvrb}
\newcommand{\VerbBar}{|}
\newcommand{\VERB}{\Verb[commandchars=\\\{\}]}
\DefineVerbatimEnvironment{Highlighting}{Verbatim}{commandchars=\\\{\}}
% Add ',fontsize=\small' for more characters per line
\usepackage{framed}
\definecolor{shadecolor}{RGB}{248,248,248}
\newenvironment{Shaded}{\begin{snugshade}}{\end{snugshade}}
\newcommand{\AlertTok}[1]{\textcolor[rgb]{0.94,0.16,0.16}{#1}}
\newcommand{\AnnotationTok}[1]{\textcolor[rgb]{0.56,0.35,0.01}{\textbf{\textit{#1}}}}
\newcommand{\AttributeTok}[1]{\textcolor[rgb]{0.13,0.29,0.53}{#1}}
\newcommand{\BaseNTok}[1]{\textcolor[rgb]{0.00,0.00,0.81}{#1}}
\newcommand{\BuiltInTok}[1]{#1}
\newcommand{\CharTok}[1]{\textcolor[rgb]{0.31,0.60,0.02}{#1}}
\newcommand{\CommentTok}[1]{\textcolor[rgb]{0.56,0.35,0.01}{\textit{#1}}}
\newcommand{\CommentVarTok}[1]{\textcolor[rgb]{0.56,0.35,0.01}{\textbf{\textit{#1}}}}
\newcommand{\ConstantTok}[1]{\textcolor[rgb]{0.56,0.35,0.01}{#1}}
\newcommand{\ControlFlowTok}[1]{\textcolor[rgb]{0.13,0.29,0.53}{\textbf{#1}}}
\newcommand{\DataTypeTok}[1]{\textcolor[rgb]{0.13,0.29,0.53}{#1}}
\newcommand{\DecValTok}[1]{\textcolor[rgb]{0.00,0.00,0.81}{#1}}
\newcommand{\DocumentationTok}[1]{\textcolor[rgb]{0.56,0.35,0.01}{\textbf{\textit{#1}}}}
\newcommand{\ErrorTok}[1]{\textcolor[rgb]{0.64,0.00,0.00}{\textbf{#1}}}
\newcommand{\ExtensionTok}[1]{#1}
\newcommand{\FloatTok}[1]{\textcolor[rgb]{0.00,0.00,0.81}{#1}}
\newcommand{\FunctionTok}[1]{\textcolor[rgb]{0.13,0.29,0.53}{\textbf{#1}}}
\newcommand{\ImportTok}[1]{#1}
\newcommand{\InformationTok}[1]{\textcolor[rgb]{0.56,0.35,0.01}{\textbf{\textit{#1}}}}
\newcommand{\KeywordTok}[1]{\textcolor[rgb]{0.13,0.29,0.53}{\textbf{#1}}}
\newcommand{\NormalTok}[1]{#1}
\newcommand{\OperatorTok}[1]{\textcolor[rgb]{0.81,0.36,0.00}{\textbf{#1}}}
\newcommand{\OtherTok}[1]{\textcolor[rgb]{0.56,0.35,0.01}{#1}}
\newcommand{\PreprocessorTok}[1]{\textcolor[rgb]{0.56,0.35,0.01}{\textit{#1}}}
\newcommand{\RegionMarkerTok}[1]{#1}
\newcommand{\SpecialCharTok}[1]{\textcolor[rgb]{0.81,0.36,0.00}{\textbf{#1}}}
\newcommand{\SpecialStringTok}[1]{\textcolor[rgb]{0.31,0.60,0.02}{#1}}
\newcommand{\StringTok}[1]{\textcolor[rgb]{0.31,0.60,0.02}{#1}}
\newcommand{\VariableTok}[1]{\textcolor[rgb]{0.00,0.00,0.00}{#1}}
\newcommand{\VerbatimStringTok}[1]{\textcolor[rgb]{0.31,0.60,0.02}{#1}}
\newcommand{\WarningTok}[1]{\textcolor[rgb]{0.56,0.35,0.01}{\textbf{\textit{#1}}}}
\usepackage{graphicx}
\makeatletter
\def\maxwidth{\ifdim\Gin@nat@width>\linewidth\linewidth\else\Gin@nat@width\fi}
\def\maxheight{\ifdim\Gin@nat@height>\textheight\textheight\else\Gin@nat@height\fi}
\makeatother
% Scale images if necessary, so that they will not overflow the page
% margins by default, and it is still possible to overwrite the defaults
% using explicit options in \includegraphics[width, height, ...]{}
\setkeys{Gin}{width=\maxwidth,height=\maxheight,keepaspectratio}
% Set default figure placement to htbp
\makeatletter
\def\fps@figure{htbp}
\makeatother
\setlength{\emergencystretch}{3em} % prevent overfull lines
\providecommand{\tightlist}{%
  \setlength{\itemsep}{0pt}\setlength{\parskip}{0pt}}
\setcounter{secnumdepth}{5}
\usepackage{draftwatermark}
\usepackage{xcolor}
\usepackage{siunitx}
\newcolumntype{d}{S[table-format=3.2]}
\usepackage{booktabs}
\usepackage{longtable}
\usepackage{array}
\usepackage{multirow}
\usepackage{wrapfig}
\usepackage{float}
\usepackage{colortbl}
\usepackage{pdflscape}
\usepackage{tabu}
\usepackage{threeparttable}
\usepackage{threeparttablex}
\usepackage[normalem]{ulem}
\usepackage{makecell}
\usepackage{xcolor}
\ifLuaTeX
  \usepackage{selnolig}  % disable illegal ligatures
\fi
\IfFileExists{bookmark.sty}{\usepackage{bookmark}}{\usepackage{hyperref}}
\IfFileExists{xurl.sty}{\usepackage{xurl}}{} % add URL line breaks if available
\urlstyle{same}
\hypersetup{
  pdftitle={Multi-asset Investment strategy Using Quantitative Models 多资产策略量化模型},
  pdfauthor={Albert Huang},
  hidelinks,
  pdfcreator={LaTeX via pandoc}}

\title{Multi-asset Investment strategy Using Quantitative Models
多资产策略量化模型}
\author{Albert Huang}
\date{2024-03-12}

\begin{document}
\frontmatter
\maketitle

{
\setcounter{tocdepth}{2}
\tableofcontents
}
\mainmatter
\hypertarget{intro-on-gtaa}{%
\chapter{INTRO ON GTAA}\label{intro-on-gtaa}}

\texttt{In\ Progress}

\hypertarget{gtaa-specificis}{%
\section{GTAA SPECIFICIS}\label{gtaa-specificis}}

\texttt{In\ Progress}

\hypertarget{multiple-assets}{%
\section{MULTIPLE ASSETS}\label{multiple-assets}}

\texttt{In\ Progress}

\hypertarget{reserach}{%
\section{RESERACH}\label{reserach}}

\texttt{In\ Progress}

\hypertarget{quick-table-of-contents-ux5febux901fux7d22ux5f15}{%
\chapter*{Quick Table of Contents
快速索引}\label{quick-table-of-contents-ux5febux901fux7d22ux5f15}}
\addcontentsline{toc}{chapter}{Quick Table of Contents 快速索引}

sss

@ref(2.1) for stock selection model

@ref(2.1) for stock selection model

\hypertarget{investment-strategy-outlook-ux6295ux8d44ux7b56ux7565ux53caux6a21ux578bux89c4ux5212}{%
\chapter{Investment Strategy Outlook
投资策略及模型规划}\label{investment-strategy-outlook-ux6295ux8d44ux7b56ux7565ux53caux6a21ux578bux89c4ux5212}}

title: ``Stock Selection Strategy Using Quantitative Models
量价选股策略模型'' subtitle: ``The Base Model: a Technical Signaling
Tool for Stock Selection 基础模型:技术指标选股工具''

\hypertarget{research-objectives-ux7814ux7a76ux76eeux6807}{%
\section{Research Objectives
研究目标}\label{research-objectives-ux7814ux7a76ux76eeux6807}}

For the purpose in discovering the potential U.S. equity investment
opportunities, we are attempting to utilize a quantitative methodology
based on a programming approach which adapts a price/volume or trend
trading strategy by constructing a model using a combination of
objective evaluation criteria, technical indicators, and fundamental
factors.

翻译

\hypertarget{considerations-on-investment-strategy-ux6295ux8d44ux7b56ux7565ux7684ux601dux8003}{%
\section{Considerations on Investment Strategy
投资策略的思考}\label{considerations-on-investment-strategy-ux6295ux8d44ux7b56ux7565ux7684ux601dux8003}}

The major investment goals are:

\begin{itemize}
\item
  Construct a stock-only portfolio with (appropriate) maximized expected
  upside returns in the long-run;
\item
  Construct the portfolio such overall and continuous downside risks are
  minimized.
\end{itemize}

\(\Rightarrow\) Investment Goal: Construct a (stock-only) portfolio with
maximum long-term risk-adjusted returns while controlling the drawdown
proactively.

翻译

In order to select a pool of U.S. stocks (from the major U.S. Indexes)
which are in consistent with the preset investment goals \& risk
preferences, we consider the following trading perspectives:

\begin{itemize}
\item
  In the world of trading, signals are \textbf{indicators} which are
  derived from various continuously changing statistics and variables,
  and guide the investors on trading directions (buy, sell, and hold).
\item
  Such signals or indicators help the traders and PMs becoming more
  informed and aware of the current market performance/conditions.
\end{itemize}

翻译

There are two major approaches to classify the indicators or to analyze
any asset, i.e.~technical analysis \& fundamental analysis:

\begin{itemize}
\item
  \textbf{Fundamental analysis} focuses on the performance of an asset
  by studying various factors/indicators, which may impact the asset
  price, such as the company's earnings, its cash flow statement and
  balance sheet, operating efficiency, corporate governance, sector
  outlook, macro trend, and etc.
\item
  Unlike the income statement modeling and industry/sector research
  conducted by the fundamental analysis, \textbf{technical analysis}
  aims to predict the future movements(patterns and trends) of certain
  indicators of an asset (mostly price and volume related); based soley
  on the historical/past characteristics of various technical
  indicators, which are (typically) displayed graphically in charts.
\end{itemize}

翻译

The core assumptions behind these two major methods are that:

\begin{enumerate}
\def\labelenumi{(\arabic{enumi})}
\item
  The fundamental analysis approach uses information (directly or
  indirectly) related to the asset itself (e.g EBITDA of the firm), and
  assumes those information were already factored into the market price
  fluctuations in time (the EMH). By identifying the historical patterns
  of these indicators or factors, and assuming they provide sufficient
  information in predicting the (future) trend, one can predict such
  indicator for the near future (e.g FY26E \$EPS). Further, utilizing
  modern finance models like the CAPM and the multi-factor models with
  Machine Learning, future asset price movements can be predicted with
  more confidence (however, not necessarily accuracy).
\item
  the assumptions for the technical analysis approach are much simpler.
  One believes that the historical up/down trend of a technical
  indicator will continue on that path or the path will reverse in the
  near-term future. In other words, technical indicators like price and
  trading volume, are assumed to move in trends or counter trends, which
  are repetitive, with certain patterns reoccurring, i.e. History will
  repeat itself (Example: momentum trading strategy and mean reversion
  strategy).
\end{enumerate}

翻译

With the research goal in building a portfolio with maximum long-term
risk-adjusted returns while controlling the maximum drawdown, we
certainly need to invest in a pool stocks with the largest winning
probability in gaining positive investment returns and with the least
amount of volatility.

By acknowledging the usefulness and limitations of fundamental and
technical analysis, we can design a quantitative model based on these
analytically methodologies to build a portfolio which best suit our
preset investment goals and risk preferences.

翻译

\begin{longtable}[t]{>{}l>{}ll}
\caption{\label{tab:unnamed-chunk-3}EMA Trading Signals (TSLA.US)}\\
\toprule
a & b & c\\
\midrule
 & 成交量类因子
\textbf{\cellcolor{gray!10}{(Volume Factor)}} & \textbf{\cellcolor{gray!10}{Accumulation/Distribution Oscillator (ADOSC); Chaikin Money Flow (CMF); Price Volume Trend (PVT); On-Balance Volume (OBV); Archer On-Balance Volume (AOBV); Elder’s Force Index (EFI); Ease of Movement (EOM); Money Flow Index (MFI); Positive Volume Index (PVI); Negative Volume Index (NVI); Price-Volume (PVOL); Price Volume Rank (PVR); Volume Profile (VP); etc.}}\\
 & 动量类因子
\textbf{(Momentum Factor)} & \textbf{Rate of Change (ROC); Awesome Oscillator (AO); Absolute Price Oscillator (APO); Bias (BIAS); Balance of Power (BOP); Commodity Channel Index (CCI); Chanda Forecast Oscillator (CFO); Center of Gravity (CG); Correlation Trend Indicator (CTI); Efficiency Ratio (ER); Elder Ray Index (ERI); Moving Average Convergence Divergence (MACD); Stochastic Oscillator (KDJ); Inertia; Relative Strength Index (RSI); Relative Strength Xtra (RSX); n-day Momentum (MOM); Psychological Line (PSL); Slope; Stochastic Momentum Index Ergodic Indicator (SMI Ergodic Indicator); Squeeze (SQZ); Squeeze Pro (SQZPRO); Stochastic Relative Strength Index (STOCHRSI); Triple Exponentially Smoothed Moving Average (TRIX); True Strength Index (TSI); etc.}\\
 & 趋势类因子
\textbf{\cellcolor{gray!10}{(Trend Factor)}} & \textbf{\cellcolor{gray!10}{Average Directional Movement Index (ADX); Archer Moving Averages Trends (AMAT); Choppiness Index (CHOP); Decay; Increasing/Decreasing; Detrend Price Oscillator (DPO); Long run/Short run; Q Stick (qstick); TTM Trend; Vortex (Vortex Indicator); and etc.}}\\
 & 波动性类因子
\textbf{(Volatility Factor)} & \textbf{Historical Volatility (HV); Implied Volatility (IV); beta (\$\textbackslash{}beta\$); Aberration; True Range (TR); Average True Range (ATR); Bollinger Bands (BBands); Mass Index (massi); Relative Volatility Index (RVI); Acceleration Bands (accbands); Elder’s Thermometer (thermos); Ulcer Index (ui); etc.}\\
Fundamental Indicators (基本面指标) & 基本面类因子 (经营效率,盈利能力,成长性和估值,现金流,财务质量) 
\textbf{\cellcolor{gray!10}{(Fundamental Factor)}} & \textbf{\cellcolor{gray!10}{Asset Turnover Ratio (ATR); Inventory Turnover Ratio (ITR); Account Receivable Turnover Ratio (ARTR); Long-term Asset Turnover Ratio (LATR), Short-term (Operating) Asset Turnover Ratio (SATR); EV/EBITDA; EV/Sales; Free Cash Flow Yield (FCF Yield); Price/Earnings Ratio (PE); Price/Book Ratio (PB); Price/Sales (PS); Return on Assets (ROA); Return on Equity (ROE); Return on Invested Capital (ROIC); Dividend Yield; Gross Margin Ratio (GMR); Operating Profit Margin Ratio (OMR); Net Profit/EBIT Margin Ratio (NPR); EBITDA Margin Ratio; Weighted Average Cost of Capital (WACC); Economic Profit (ROIC-WACC); and etc.}}\\
 & 风格因子 
\textbf{(Other Style Factor)} & \textbf{Size; Dividend; Sentiment; etc.}\\
Portfolio Indicators (组合类指标) & 投资策略因子
\textbf{\cellcolor{gray!10}{(Strategy \& Portfolio Factor)}} & \textbf{\cellcolor{gray!10}{Information Ratio; Maximum Drawdown (MDD); etc.}}\\
\bottomrule
\multicolumn{3}{l}{\rule{0pt}{1em}\textit{Note: }}\\
\multicolumn{3}{l}{\rule{0pt}{1em}Output 2.3.2.b: Last 8 rows are shown.}\\
\multicolumn{3}{l}{\rule{0pt}{1em}\textsuperscript{1} Includes tsignal4, tsignal5, tsignal6, tsignal6a and tsignal6b.}\\
\end{longtable}

\hypertarget{considerations-on-model-designs-ux6a21ux578bux8bbeux8ba1ux7684ux601dux8003}{%
\section{Considerations on Model Designs
模型设计的思考}\label{considerations-on-model-designs-ux6a21ux578bux8bbeux8ba1ux7684ux601dux8003}}

In achieving the ultimate investment goal of building a `high-return and
low-risk' stock-only portfolio, we are aiming to build a quantitative
model with the most effective components.

There are countless relative metrics and factors evaluating an asset or
a firm, whether the particular indicator/factor is \textbf{significant}
to the price movements is a question in much more depth.

翻译

To have `meaningful' model components, we will divide and concur the
research objectives by first selecting stocks with near-term investment
opportunities (the \texttt{base\ model}). Following, we will then filter
the selected list of stock more rigorously so the positive investment
return are more `certain' for the longer-term with minimal downside risk
(the \texttt{full\ model}).

In other words, the \texttt{base\ model} will input the
\textbf{significant} technical indicators to select a list of `possible
winning' U.S stocks (with potential investment opportunities).
Following, the \texttt{complete\ model} (\emph{Section 5}) will include
the \textbf{significant} fundamental indicators/factors to further
filter from the list of selected stocks, and keep the ones with `higher
certainty' of long-term risk-adjusted (investment) returns.

翻译

Besides selecting stocks based on appropriate and significant
indicators/factors with the greatest investment return potentials, the
model will also include \textbf{subjective filters} to suit specific
investment needs/goals (example: able to buy and hold, hot areas of
investment interest), risk preferences, and compliance requirements
(example: invest in market cap \(>\$7B\))。

翻译 过滤条件是主观设定的,同时可以根据其他投资需求与风险偏好进行筛选。

A \textbf{technical indicator} is basically a mathematical
representation and manipulation of the basic raw trading data and
statistics of an asset (e.g.~adj close price, trading volume, 52-week
high and low, etc.). In other words, a technical indicator is usually a
derivation of the raw trading statistics and is designed to
represent/signal certain market behavior.

\begin{itemize}
\tightlist
\item
  Traders typically view the technical indicators as tools on a trading
  chart to analyze the market trend in a clearer manner. In trading, a
  technical indicator is like a financial compass, which provides
  certain market behavior insights (overbought, oversold, potential
  reversal, range-bound, etc), so investment opportunities and risks can
  be identified more clearly and intuitively. Traders and PMs can
  utilize a combination of technical indicators to make informed
  real-time trading decisions with more confidence.
\end{itemize}

\textbf{here}

One of the ultimate goals in any type of security analysis is to
understand/predict the `direction' of the asset's future price
movements. There are various technical indicators that traders use to
predict/deduce future price levels (\emph{Section 2.2}).

\begin{itemize}
\item
  One of the common technical trading strategies is momentum trading,
  which is also the core assumptions we made in building the stock
  selection model. A stock that has been rising is said to have positive
  momentum while a stock that has been crashing is said to have negative
  momentum. Momentum is an investing factor that aims to benefit from
  the ongoing trend of a stock or asset.
\item
  Momentum trading centers on buying and selling assets following their
  recent performance trends. Traders using this strategy believe that
  assets moving strongly in a direction will continue to do so. They aim
  to capitalize on this momentum before it fades.
\end{itemize}

*Trend followers believe that stocks moving up or down will continue on
that path.

In contrast, mean reversion suggests that assets overreact and will
eventually return to their mean values.Momentum trading thrives in
markets exhibiting strong trends. It's less effective in sideways or
highly volatile markets. Therefore, identifying the right market
conditions is critical for success. Sudden market reversals can quickly
erode gains. Hence, effective risk management is essential.

Technical Indicators do not follow a general pattern, meaning, they
behave differently with every security. What can be a good indicator for
a particular security, might not hold the case for the other. Thus,
using a technical indicator requires jurisprudence coupled with good
experience.

\hypertarget{base-model-construction-ux57faux7840ux6a21ux578bux642dux5efa}{%
\chapter{(Base) Model Construction
基础模型搭建}\label{base-model-construction-ux57faux7840ux6a21ux578bux642dux5efa}}

\hypertarget{intro-on-base-model-design-and-construction-ux6a21ux578bux8bbeux8ba1ux642dux5efaux6982ux8981}{%
\section{Intro on (Base) Model Design and Construction
模型设计搭建概要}\label{intro-on-base-model-design-and-construction-ux6a21ux578bux8bbeux8ba1ux642dux5efaux6982ux8981}}

The model designing to select stocks is heavily dependent on the
portfolio strategy and risk preferences. To reiterate, the
\texttt{base\ model} will select stocks signaling near-term investment
opportunities based on a combination of \textbf{technical indicators}.

Over 6,000 stocks currently listed in the U.S. security market, however,
some may not be appropriate for the portfolio (style, risk, compliance,
etc.). For instance, a stock went public one year ago with a market cap
of \$10B does not satisfy the investment style and risk reference. To
save computing power, the model will subjectively filter out these
`inappropriate' stocks in advance to any machine selection processes
based on indication signals.

翻译

More specifically:

\begin{itemize}
\tightlist
\item
  Step 1: Fetch market data:
\end{itemize}

The model will gather up-to-date raw trading data and statistics (open,
close, volume etc.) for stocks currently listed on the NYSE and NASDAQ
(U.S. stocks listed on the following exchanges: PHLX, MS4X, BSE, CHX,
and NSX are excluded in this study).

翻译

\begin{itemize}
\tightlist
\item
  Step 2: Drop `nonviable' stocks (Portfolio-specific Filters):
\end{itemize}

First, the \textbf{subjective filters} eliminate the `inappropriate'
stocks, which are the ones will not be considered as a viable investment
option; because of the unmatched management styles, internal risk
management guidelines, risk preference, investor/client investment
preferences/risk tolerance, etc. (subjective: the eliminated securities
are due to subjective preset investment goals, management styles or risk
management guidelines; but may be stocks could bring substantial future
returns).

翻译

\begin{itemize}
\tightlist
\item
  Step 3: Calculate the indicators (Technical Indicators Construction):
\end{itemize}

There are countless of technical indicators, and we will first pick a
pool of indicators we deem fit in characterizing the U.S equity price
performances. Following, with a programmatic approach, the model
computes the indicator value(s), and produces any meaningful graphs. In
terms of trading signals, we will set conventional signal parameters,
where further investment-specific setting adjustments may become
necessary. Moreover, we will use a handful of widely-traded stocks to
test whether the program for each technical indicators perform by
providing trading signals as intended (not a test on signal accuracy).

翻译

\begin{itemize}
\tightlist
\item
  Step 4: Select a combination of significant technical indicators
  (Technical Indicator Effectiveness Ranking)
\end{itemize}

The characteristics and preset assumptions of the technical indicators,
meaning they behave differently for each market with every security. For
instance, one can be an effective indicator in terms of signaling
accuracy for a stock in a more traded sector, may not hold the case for
another stock which is less liquid. Similar applies to the indicator
parameters, where different parameter settings may lead to opposite
trading signals during different periods. Whether the following
indicators are effective and accurate is questionable. Thus, the model
utilizes a ranking system to objectively determine their effectiveness,
i.e.~\textbf{significant} indicators. Using technical indicators in
reaching a profitable trading decision requires jurisprudence coupled
with investment experience.

The model methodology is to use the full dataset with all 5,400+ stocks
currently listed on NYSE and NASDAQ, to test and rank the technical
indicators in terms of the accuracy in giving the correct trading
signals (1,997 NYSE, 3,433 NASDAQ as of Feb 22, 2024). For example, say
the RSI accuracy is 70\% and No.1 among all technical indicators, by
giving the correct next-day trading signals for 3,780+ stocks week-long;
which its signal accuracy is higher than any other indicators. With a
programmatic approach, the effectiveness evaluation processes produce an
accuracy-based ranking for all input technical indicators.

翻译

However, using a technical-only based model emerges a practical and
tricky issue, where one \textbf{significant} indicator may disagree with
another \textbf{significant} indicator on trading signals. A subjective
set of indicator parameters may also cause similar mixed-signal issues.
In practice, analysts and PMs reference a combination of the technical
indicators, along with other security analysis methods (fundamental,
quant, etc.) before arriving at a trading decision (Buy/Sell/Hold).

Moreover, investment strategies usually vary for different fund
products. To have the stock selection model becoming more adaptive and
customizable for various investment needs; for example, to be able to
output two separate top-30 stock lists for a buy-and-hold strategy and a
long-short risk-neutral strategy. Therefore, the model not only need to
be adaptive to input different sets of significant technical indicators
(example: 2 sets of technical indicators to \textbf{signal for
shorter-and-longer-term returns}), but also for numerous combinations of
parameter settings.

翻译

More specifically (for Step 4), the model first utilizes the full (U.S.
stock) dataset to test and rank the effectiveness of all input technical
indicators, and produces two combinations of significant technical
indicators, one set for shorter-term (intraday, next day) signals and
one set for longer-term signals.

Regarding the parameter settings for the technical indicators, the
process in determining the `best' setting is relatively subjective. It
requires manual adjustments for each (of them who needs a parameter
input), in which different settings at different sample (testing)
periods may result differently. Most importantly, trading signals
directly influenced by the parameter settings; and the investment
styles/goals also have direct impact on the settings (e.g.~an aggressive
strategy, a trend-following strategy, and an EIF strategy certainly
requires different parameter settings, and similar applies to initiate
trades on the left or right).

翻译

\begin{itemize}
\tightlist
\item
  Step 5: Stock Selection \& Trading Signals -- the Base Model By
  finishing the process of datasets inputs, and the elimination of
  stocks on portfolio-specific (`nonviable') filters, the (base) stock
  selection model can now fetch the up-to-date stock trading statistics
  through the chosen combinations of significant technical indicators.
\end{itemize}

By design, every technical indicator generates a trading signal based on
the given parameter settings for each stock. Following, the model counts
the number of trading signals generated for each stock.

An aggregate signaling rule needs to be established in advance for the
model to generate the trading recommendation, however, this setting can
be modified easily to suit specific investment needs and the
continuously changing market landscape.

翻译

For example, the PMs can set the rule with a \textbf{(theoretical)
winning-probability threshold} of 75\%, and say from the model output,
\texttt{stock\ A} generated 11 \texttt{short\ position} signals from a
total of 14 significant technical indicators, which
\(\frac{11}{14}>75\%\). The stock-selection model which indicated that
\texttt{stock\ A} has a(n) aggregate \emph{SELL} signal as of today.
Obviously, with the subjective setting of the winning probability
threshold, the programming approach of the model can automate the
tedious and heavy computing process, and produces a list of \texttt{BUY}
stocks, and a list of \texttt{SELL} stocks.

Furthermore, to suit various investment goals/needs like multiple
strategies for different portfolios; the model input can easily be
modified into two combinations of technical indicators. As an example,
besides generating trading recommendations (BUY/SELL/HOLD) for each
stock, the model can further indicate/signal whether which stocks have
the investment potential (either direction) for a longer-term (Note:
such additional recommendation may be due to the extra combination of
technical indicators and/or different indicator settings).

Above summarized the complete processes in detail, to construct an
adaptative U.S. stock selection model using only technical indicators
(i.e.~\texttt{the\ base\ model}) and to generate trading recommendations
accordingly.

翻译

Note: We believe technical indicators are relatively objective as they
are derivations of market data and statistics. Therefore, by theory,
such model should include the less randomness and bias in trading
recommendations. To control portfolio drawdowns, while seeking for
higher risk-adjusted returns, we purposely divide the model into two
parts and introduce model factors/indicators stepwise. Other types of
analytical data and statistics include somewhat subjectivity, however,
the complete stock selection model (i.e.~\texttt{the\ full\ model})
(\emph{Section 5}) will accept significant fundamental, growth/value,
volatility, and emotion factors.

翻译

\begin{itemize}
\tightlist
\item
  Step 6: Base Model Back-test and RM Additions (Portfolio Simulation
  {[}Setting: Base Model, Allocation: Equal Weight{]})
\end{itemize}

After gathered the recommended list of stocks to invest generated by the
\texttt{base\ model}, the program is set to evaluate the model
performance. A standard back-test is then performed, in which the
portfolio is constituted by the model-selected U.S. stocks. Trades are
set to execute in accordance with the model-generated trading signals
(BUY/SELL/HOLD), and the set theoretical winning-probability threshold
(the aggregate signaling rule). For the simplicity and the purpose in
controlling variables, the asset allocation rule for this simulated
portfolio is set to be equal weight. Finally, the program computes the
investment return of the simulated portfolio in percentages.

In constructing a model to select U.S. equities, and assess the strategy
performance, we assumed a stock-only portfolio. In practice, we need to
consider the risk exposure. Effective and proactice risk management is
essential for a stock-heavy portfolio. For instance, depends on the
overall volatility from the selected stocks, the PMs can invest in ETFs
or appropriate commodities or keep a higher cash or cash equivalent
asset to hedge.

翻译

\textless insert.png\textgreater{} {[}model design flow chart{]}

Above has explained in great details about the expectations of the price
volume strategy, the philosophy in designing the stock selection model
and the necessary steps in constructing such model with a programming
approach. A `technical-only' quantitative model is rare today, and
without much reference to follow, subjective and fundamental factors are
usually included in constructing such portfolio. Therefore, the report
has exhibited the assumptions, subjective decisions, and all influential
elements for the designing and modeling processes in detail.

Following the model flow chart, the sections below will illustrate the
the \texttt{base\ model}, include computations, visualizations,
analyses, simulations and etc.

翻译

\emph{\textcolor{red}{Code Hidden 代码已隐藏}}

\begin{table}[H]
\centering
\caption{\label{tab:unnamed-chunk-9}Cleaned Data Glance: NYSE and NASDAQ Stocks}
\centering
\resizebox{\ifdim\width>\linewidth\linewidth\else\width\fi}{!}{
\begin{tabular}[t]{ll>{}lrrrr>{}r>{}r>{}r}
\toprule
  & date & symbol & adjusted & close & high & low & open & volume & market\_cap\\
\midrule
\cellcolor{gray!10}{2000} & \cellcolor{gray!10}{2024-02-22} & \textbf{\cellcolor{gray!10}{MSFT}} & \cellcolor{gray!10}{411.6} & \cellcolor{gray!10}{411.6} & \cellcolor{gray!10}{412.8} & \cellcolor{gray!10}{408.6} & \textbf{\cellcolor{gray!10}{410.2}} & \textbf{\cellcolor{gray!10}{27,009,900}} & \textbf{\cellcolor{gray!10}{3,027,755,746,755}}\\
2001 & 2024-02-22 & \textbf{NVDA} & 785.4 & 785.4 & 785.8 & 742.2 & \textbf{750.2} & \textbf{86,510,000} & \textbf{1,967,525,024,414}\\
\cellcolor{gray!10}{2002} & \cellcolor{gray!10}{2024-02-22} & \textbf{\cellcolor{gray!10}{TSLA}} & \cellcolor{gray!10}{197.4} & \cellcolor{gray!10}{197.4} & \cellcolor{gray!10}{198.3} & \cellcolor{gray!10}{191.4} & \textbf{\cellcolor{gray!10}{194.0}} & \textbf{\cellcolor{gray!10}{92,739,500}} & \textbf{\cellcolor{gray!10}{636,098,096,289}}\\
2003 & 2024-02-23 & \textbf{AAPL} & 182.5 & 182.5 & 185.0 & 182.2 & \textbf{185.0} & \textbf{45,074,500} & \textbf{2,820,154,184,738}\\
\cellcolor{gray!10}{2004} & \cellcolor{gray!10}{2024-02-23} & \textbf{\cellcolor{gray!10}{AMZN}} & \cellcolor{gray!10}{175.0} & \cellcolor{gray!10}{175.0} & \cellcolor{gray!10}{175.8} & \cellcolor{gray!10}{173.7} & \textbf{\cellcolor{gray!10}{174.3}} & \textbf{\cellcolor{gray!10}{59,662,900}} & \textbf{\cellcolor{gray!10}{1,802,629,270,728}}\\
2005 & 2024-02-23 & \textbf{GOOG} & 145.3 & 145.3 & 146.0 & 144.8 & \textbf{145.0} & \textbf{14,508,400} & \textbf{1,733,681,490,697}\\
\cellcolor{gray!10}{2006} & \cellcolor{gray!10}{2024-02-23} & \textbf{\cellcolor{gray!10}{META}} & \cellcolor{gray!10}{484.0} & \cellcolor{gray!10}{484.0} & \cellcolor{gray!10}{494.4} & \cellcolor{gray!10}{482.4} & \textbf{\cellcolor{gray!10}{488.0}} & \textbf{\cellcolor{gray!10}{17,861,100}} & \textbf{\cellcolor{gray!10}{1,241,690,132,758}}\\
2007 & 2024-02-23 & \textbf{MSFT} & 410.3 & 410.3 & 415.9 & 409.0 & \textbf{415.7} & \textbf{16,284,800} & \textbf{3,027,755,746,755}\\
\cellcolor{gray!10}{2008} & \cellcolor{gray!10}{2024-02-23} & \textbf{\cellcolor{gray!10}{NVDA}} & \cellcolor{gray!10}{788.2} & \cellcolor{gray!10}{788.2} & \cellcolor{gray!10}{823.9} & \cellcolor{gray!10}{775.7} & \textbf{\cellcolor{gray!10}{807.9}} & \textbf{\cellcolor{gray!10}{82,711,200}} & \textbf{\cellcolor{gray!10}{1,967,525,024,414}}\\
2009 & 2024-02-23 & \textbf{TSLA} & 192.0 & 192.0 & 197.6 & 191.5 & \textbf{195.3} & \textbf{78,670,300} & \textbf{636,098,096,289}\\
\bottomrule
\multicolumn{10}{l}{\rule{0pt}{1em}\textit{Note: }}\\
\multicolumn{10}{l}{\rule{0pt}{1em}Output 2.1.a: Last 10 rows are shown.}\\
\end{tabular}}
\end{table}

\hypertarget{portfolio-specific-subjective-indicators-ux5176ux4ed6ux7ec4ux5408ux76f8ux5173ux8fc7ux6ee4ux6307ux6807}{%
\section{Portfolio-specific Subjective Indicators
其他组合相关过滤指标}\label{portfolio-specific-subjective-indicators-ux5176ux4ed6ux7ec4ux5408ux76f8ux5173ux8fc7ux6ee4ux6307ux6807}}

With over 6,000 stocks currently lists on the U.S. stock exchanges, the
model will first eliminate the `inappropriate' stocks, i.e.~the stocks
will not be considered as viable investment opportunities.

The following are the areas we take into considerations in eliminating
the `inappropriate' stocks:

\begin{itemize}
\tightlist
\item
  Investment needs/preset goals \& risk preferences (specific to the
  managed portfolio)
\item
  Management styles (avoid style drift or inconsistent investment
  styles)
\item
  Liquidity requirements
\item
  Client/Investor needs \& risk tolerance
\item
  Internal RM guidelines \& compliance requirements
\end{itemize}

To reiterate, the eliminated stocks are dropped (from the potential
investment list) due to the above subjective reasons; in no means these
stocks will necessarily bring negative investment returns, some of them
could bring substantial (positive) future returns.

翻译

Thus, we set the following \textbf{subjective indicators} to meet the
strategy needs (this list may be modified with the fast-changing market
landscape):

\begin{enumerate}
\def\labelenumi{(\arabic{enumi})}
\tightlist
\item
  \(\text{Market Cap}> \$ 7 \text{ billion}\)\\
\item
  \(\text{Average Weekly Volume}> 1 \text{ million}\)\\
\item
  \(\text{Daily Return} \nless -15\%\)\\
\item
  \(\text{Annualized Volatility}\ngtr 100\%\).
\end{enumerate}

翻译.

\hypertarget{section}{%
\subsection{\texorpdfstring{\textcolor{blue}{Subjective indicator No.1: Market Cap > \$ 7 billion}}{}}\label{section}}

The program gathers the most up-to-date (value at close on the last
trading day) \emph{market cap} (\texttt{market\_cap}) for each company,
the stocks with a latest market cap smaller than 7 billion USD will be
eliminated (from the stock selection list).

A market cap less than \$7 billion, indicates the stock is more growth
in nature, and with less analyst coverage. `Growth stocks' have the
growth potential, however tend to be more volatile. To match the
investment goals/portfolio needs, satisfy internal RM and compliance
guidelines, and actively controlling the drawdown and down-side risks;
the model therefore in advance, eliminates these stocks and label them
as `inviable' investment options.

翻译

\emph{\textcolor{red}{Code Hidden 代码已隐藏}}

\begin{verbatim}
## character(0)
\end{verbatim}

Above output shows the \texttt{cap\_remove\_list}, i.e.~the stocks
removed from the \texttt{stockdata} dataset due to its \emph{market cap}
was smaller than 7 billion USD (\(7\times 10^9\)) some time during last
year, to meet internal RM guidelines, and actively manage the potential
downside risks and return fluctuation (drawdown) issues.

翻译

\hypertarget{section-1}{%
\subsection{\texorpdfstring{\textcolor{blue}{Subjective indicator No.2: Average Weekly Volume > 1 million}}{}}\label{section-1}}

The program first computes the average weekly (trading) \texttt{Volume},
the stock with an \texttt{average\ volume} \textbf{smaller than 1
million} some time during the last year, will be eliminated (from the
stock list).

A daily (average) trading volume of 1 million or less, indicates the
stock was traded inactively in relative terms. To satisfy internal RM
and compliance guidelines, and avoid any liquidity risks; the model
therefore in advance, eliminates these stocks and label them as
`inviable' investment options.

翻译.

\emph{\textcolor{red}{Code Hidden 代码已隐藏}}

\begin{longtable}[t]{rll>{}r}
\caption{\label{tab:unnamed-chunk-13}Average Weekly Trading Volume}\\
\toprule
Year & Week & Symbol & Average Volume\\
\midrule
\cellcolor{gray!10}{2024} & \cellcolor{gray!10}{07} & \cellcolor{gray!10}{MSFT} & \textbf{\cellcolor{gray!10}{22707120}}\\
2024 & 07 & NVDA & \textbf{52704940}\\
\cellcolor{gray!10}{2024} & \cellcolor{gray!10}{07} & \cellcolor{gray!10}{TSLA} & \textbf{\cellcolor{gray!10}{99093300}}\\
2024 & 08 & AAPL & \textbf{48140500}\\
\cellcolor{gray!10}{2024} & \cellcolor{gray!10}{08} & \cellcolor{gray!10}{AMZN} & \textbf{\cellcolor{gray!10}{50402800}}\\
2024 & 08 & GOOG & \textbf{18164575}\\
\cellcolor{gray!10}{2024} & \cellcolor{gray!10}{08} & \cellcolor{gray!10}{META} & \textbf{\cellcolor{gray!10}{17619875}}\\
2024 & 08 & MSFT & \textbf{21558425}\\
\cellcolor{gray!10}{2024} & \cellcolor{gray!10}{08} & \cellcolor{gray!10}{NVDA} & \textbf{\cellcolor{gray!10}{77183575}}\\
2024 & 08 & TSLA & \textbf{94949900}\\
\bottomrule
\multicolumn{4}{l}{\rule{0pt}{1em}\textit{Note: }}\\
\multicolumn{4}{l}{\rule{0pt}{1em}Output 2.2.a: Last 10 rows are shown.}\\
\end{longtable}

\begin{verbatim}
## character(0)
\end{verbatim}

Above output shows the \texttt{volume\_remove\_list}, i.e.~the stocks
removed from the \texttt{stockdata} dataset due to its
\texttt{average\ volume} was smaller than 1 million (\(1\times 10^6\))
some time during last year, to avoid potential liquidity issues.

翻译

\hypertarget{section-2}{%
\subsection{\texorpdfstring{\textcolor{blue}{Subjective indicator No.3: Daily Return $\nless -15\%$}}{}}\label{section-2}}

The program first computes the \emph{daily return}
(\texttt{price\_pctchange}), i.e.~the \emph{daily price}
(\texttt{adjusted}) change in percentages for each stock. Any stock with
a `price drop' of more than 15\% in any trading day during the last year
will be eliminated (from the stock selection list). FYI, the program
utilizes the daily \emph{adjusted close price} (\texttt{adjusted}) as
the \emph{daily price} for each stock inputs to the model, which they
are adjusted for any dilutions to the shares, i.e.~accounts for any
dividend distributions and applicable share splits.

A daily price change of -15\% indicates extreme short-term price down
movement. To proactively manage (max) drawdown for our portfolio, the
model therefore in advance, eliminates these stocks, which exhibited
significant short-term downside risks during last year.

翻译

\emph{\textcolor{red}{Code Hidden 代码已隐藏}}

\begin{table}[H]
\centering
\caption{\label{tab:unnamed-chunk-17}Price Percentage Change (1-day ROC)}
\centering
\resizebox{\ifdim\width>\linewidth\linewidth\else\width\fi}{!}{
\begin{tabular}[t]{ll>{}r>{}r}
\toprule
Symbol & Date & Adj Close & Price Change \%\\
\midrule
\cellcolor{gray!10}{MSFT} & \cellcolor{gray!10}{2024-02-22} & \textbf{\cellcolor{gray!10}{411.65}} & \textbf{\cellcolor{gray!10}{2.3547}}\\
NVDA & 2024-02-22 & \textbf{785.38} & \textbf{16.4009}\\
\cellcolor{gray!10}{TSLA} & \cellcolor{gray!10}{2024-02-22} & \textbf{\cellcolor{gray!10}{197.41}} & \textbf{\cellcolor{gray!10}{1.3554}}\\
AAPL & 2024-02-23 & \textbf{182.52} & \textbf{-1.0034}\\
\cellcolor{gray!10}{AMZN} & \cellcolor{gray!10}{2024-02-23} & \textbf{\cellcolor{gray!10}{174.99}} & \textbf{\cellcolor{gray!10}{0.2349}}\\
GOOG & 2024-02-23 & \textbf{145.29} & \textbf{-0.0207}\\
\cellcolor{gray!10}{META} & \cellcolor{gray!10}{2024-02-23} & \textbf{\cellcolor{gray!10}{484.03}} & \textbf{\cellcolor{gray!10}{-0.4320}}\\
MSFT & 2024-02-23 & \textbf{410.34} & \textbf{-0.3182}\\
\cellcolor{gray!10}{NVDA} & \cellcolor{gray!10}{2024-02-23} & \textbf{\cellcolor{gray!10}{788.17}} & \textbf{\cellcolor{gray!10}{0.3552}}\\
TSLA & 2024-02-23 & \textbf{191.97} & \textbf{-2.7557}\\
\bottomrule
\multicolumn{4}{l}{\rule{0pt}{1em}\textit{Note: }}\\
\multicolumn{4}{l}{\rule{0pt}{1em}Output 2.2.b: Last 10 rows are shown.}\\
\end{tabular}}
\end{table}

\emph{\textcolor{red}{Code Hidden 代码已隐藏}}

\begin{verbatim}
## character(0)
\end{verbatim}

Above output shows the \texttt{prcpct\_remove\_list}, i.e.~the stocks
removed from the \texttt{stockdata} dataset due to its \emph{daily
return} was smaller than \(-15\%\), some time during last year, to
manage short-term drawdown.

翻译

\hypertarget{section-3}{%
\subsection{\texorpdfstring{\textcolor{blue}{Subjective indicator No.4: Annualized volatility $\ngtr 100\%$}}{}}\label{section-3}}

The program first computes the (implied) \emph{Annualized Volatility}
from the \emph{daily return}s (\texttt{price\_pctchange}) for each
stock, using
\(\text{Annualized Volatility}=\Delta(\text{Daily Price})\cdot \sqrt{252}\),
assuming 252 trading days every year. Any stock with an implied annual
volatility of more than 100\% will be removed from the selection
(process).

Implied volatility (\texttt{annual\_volatility}) greater than 100\%
indicates extreme volatile price movements at some point during the last
year. To actively manage potential short-term and long-term (portfolio)
risks, and utilizing an equilibrium investment strategy; with the
potential volatile stocks are exhibited, they will be eliminated in
advance for that purpose.

翻译

\emph{\textcolor{red}{Code Hidden 代码已隐藏}}

\begin{table}[H]
\centering
\caption{\label{tab:unnamed-chunk-21}Implied Annualized Volatility}
\centering
\resizebox{\ifdim\width>\linewidth\linewidth\else\width\fi}{!}{
\begin{tabular}[t]{l>{}rl>{}r}
\toprule
Date & Adj Close & Symbol & Annual Volatility \%\\
\midrule
\cellcolor{gray!10}{2024-02-22} & \textbf{\cellcolor{gray!10}{411.65}} & \cellcolor{gray!10}{MSFT} & \textbf{\cellcolor{gray!10}{37.3792}}\\
2024-02-22 & \textbf{785.38} & NVDA & \textbf{260.3559}\\
\cellcolor{gray!10}{2024-02-22} & \textbf{\cellcolor{gray!10}{197.41}} & \cellcolor{gray!10}{TSLA} & \textbf{\cellcolor{gray!10}{21.5170}}\\
2024-02-23 & \textbf{182.52} & AAPL & \textbf{-15.9287}\\
\cellcolor{gray!10}{2024-02-23} & \textbf{\cellcolor{gray!10}{174.99}} & \cellcolor{gray!10}{AMZN} & \textbf{\cellcolor{gray!10}{3.7282}}\\
2024-02-23 & \textbf{145.29} & GOOG & \textbf{-0.3279}\\
\cellcolor{gray!10}{2024-02-23} & \textbf{\cellcolor{gray!10}{484.03}} & \cellcolor{gray!10}{META} & \textbf{\cellcolor{gray!10}{-6.8575}}\\
2024-02-23 & \textbf{410.34} & MSFT & \textbf{-5.0518}\\
\cellcolor{gray!10}{2024-02-23} & \textbf{\cellcolor{gray!10}{788.17}} & \cellcolor{gray!10}{NVDA} & \textbf{\cellcolor{gray!10}{5.6392}}\\
2024-02-23 & \textbf{191.97} & TSLA & \textbf{-43.7452}\\
\bottomrule
\multicolumn{4}{l}{\rule{0pt}{1em}\textit{Note: }}\\
\multicolumn{4}{l}{\rule{0pt}{1em}Output 2.2.c: Last 10 rows are shown.}\\
\end{tabular}}
\end{table}

\emph{\textcolor{red}{Code Hidden 代码已隐藏}}

\begin{verbatim}
## [1] "META" "NVDA"
\end{verbatim}

Above output shows the \texttt{volat\_remove\_list}, i.e.~the stocks
removed from the \texttt{stockdata} dataset due to its \emph{annualized
volatility} (\texttt{annual\_volatility}) was greater than \(100\%\),
some time during last year, to actively control unusual return
fluctuations and manage downside risks.

翻译

\emph{\textcolor{red}{Code Hidden 代码已隐藏}}

\begin{table}[H]
\centering
\caption{\label{tab:unnamed-chunk-24}Stocks Removed by the Subjective Indicators}
\centering
\resizebox{\ifdim\width>\linewidth\linewidth\else\width\fi}{!}{
\begin{tabu} to \linewidth {>{\raggedright}X>{}l}
\toprule
  & Symbol\\
\midrule
\cellcolor{gray!10}{{}[1,]} & \textbf{\cellcolor{gray!10}{META}}\\
{}[2,] & \textbf{NVDA}\\
\bottomrule
\multicolumn{2}{l}{\rule{0pt}{1em}\textit{Note: }}\\
\multicolumn{2}{l}{\rule{0pt}{1em}Output 2.2.c: Last 10 rows are shown.}\\
\multicolumn{2}{l}{\rule{0pt}{1em}\textsuperscript{*} In no means these removed stocks will bring negative investment returns, some of them may bring substantial (positive) future returns.}\\
\end{tabu}}
\end{table}

In summary, above are the programming steps of the selection model, in
eliminating the `inappropriate' stocks; to suit specific investment
needs/goals, portfolio risk preferences, and meet other internal RM
guidelines and requirements. The 4 \textbf{subjective indicators} have
removed a total \textbf{6} U.S. stocks (see the
\texttt{remove\_stock\_list} output below) from the potential stock
selection list for further modeling and analyzing (\textbf{constantly
updating by model design, result as of 2024.03.09}).

\begin{itemize}
\tightlist
\item
  To reemphasize, the stocks on the \texttt{remove\_stock\_list} were
  eliminated subject to the preset \textbf{subjective indicators} listed
  above. This particular list of indicators are dynamic, which may be
  adjusted (added/subtracted/modified) to fulfill investment needs based
  on the changing market landscape. For practical portfolio
  constructions, more \textbf{subjective indicators} should be added,
  for example, stocks on the government's (so called) `Entity List',
  should be eliminated to meet compliance (e.g.~Lockheed Martin
  {[}LMT.NYSE{]}, RTX Corporation {[}RTX.NYSE{]}, etc).
\end{itemize}

翻译.

\hypertarget{technical-indicators-construction-ux6280ux672fux6307ux6807ux642dux5efa}{%
\chapter{Technical Indicators Construction
技术指标搭建}\label{technical-indicators-construction-ux6280ux672fux6307ux6807ux642dux5efa}}

As previously defined, technical analysis is the recording of the actual
trading history, to identify potential patterns/trends with the
assumption that similiar behavior will repeat in the future. A
\textbf{technical indicator} is basically a mathematical representation
and manipulation of the basic historical raw trading data and statistics
of an asset.

The following in \(\S 2.3\) exhibits the computation results,
graphs/charts, and generated trading signals based on subjectively set
parameters for each of the \textbf{12} technical indicators. Whether one
tech indicator is effective is evaluated in \(\S 2.4\). Model simulation
for a portfolio constructed of the selected stocks will be performed and
analyzed in \(\S 3.1\).

Note: For all technical indicators utilized for this stock selection
strategy, only the traditional (type of) technical analysis was applied;
and there were no Machine Learning or AI enhanced directly for any
indicators, nor any ML-driven evaluation processes included in selecting
the significant indicators (in \(\S 2.4\)).

\begin{itemize}
\tightlist
\item
  For example: the \textbf{RSI} is the one of the most widely used
  trading indicators to detect potential oversold and overbought
  signals. As one of the classical trading tools, \textbf{RSI} has its
  defects. \textbf{RSI} tends of under or over react in sudden market
  shifts. Nowadays, quant analysts and traders often utilizes the LSTM
  (Long short-term memory) framework, one of the RNN (Recurrent Neural
  Network) Deep Learning Models to enhance the \textbf{RSI} performance,
  and applies Bayesian Optimization on the parameter settings.
\end{itemize}

翻译.

Starting at over 5,600+ stocks listed on NYSE and NASDAQ, and with
eliminations by the \textbf{subjective indicators}, about \textbf{xx}
U.S. stocks are input for technical indicators to filter and further
model processes.

In general, the program is set to perform computations and generate
tradings signals accordingly for each input stock; without occupying the
majority of report spaces, all outputs (computation summaries/tables,
charts/graphs) are therefore suppressed (A warning like below is
provided for each suppressed output). For each model step, the report
only shows the program output for Tesla, Inc.~(\texttt{TSLA.US}); the
same applies for the remaining of the report. Analysts may construct a
web ui database program, so the PMs and analysts are able to check the
indicator valuations, and their graphs or charts for the most up-to-date
selection of stocks.

\textless Insert a chart with all indicators (sig or not sig) and
character them as type of indicators mom volume etc\textgreater{}

翻译.

\hypertarget{technical-indicator-1-simple-moving-avaerage-sma}{%
\section{Technical Indicator 1: Simple Moving Avaerage
(SMA)}\label{technical-indicator-1-simple-moving-avaerage-sma}}

(Includes \texttt{tsignal1}, \texttt{tsignal2}, and \texttt{tsignal3})

翻译

\begin{itemize}
\tightlist
\item
  Intro- SMA:
\end{itemize}

Commonly, traders intend to observe the average stock prices of the last
number of trading days, is often defined as the moving average price or
the rolling average price. An n-day \textbf{simple moving average}
price, or the \textbf{n-day SMA} price, which refers to the
\textbf{arithmetic} average or the simple average of the stock prices
(the model uses the \emph{adjusted close} prices) for the past n
consecutive trading days. The SMA lines tend to smooth out volatility or
price variations, and makes the visualization of price trends more clear
and intuitive.

Mathematically defined as:

\[\text{SMA}_t=\frac{P_t+...+P_{t-n+1}}{n}=\frac{\sum^{t}_{i=t-n+1}P_i}{n}\]

The program uses the formula shown above to compute the \textbf{simple
moving average (SMA)} values for \textbf{all stocks} in the model at
various parameter settings, and creates appropriate related
visualizations. Following, the \textbf{SMA} trading signals will be
constructed.

翻译

\textcolor{red}{Individual stock outputs are suppressed, only TSLA.US related results are shown for illustration purposes.个股输出数据及图表已隐藏,报告仅展示特斯拉(TSLA.US)相关结果。}

\emph{\textcolor{red}{Code Hidden 代码已隐藏}}

\begin{table}[H]
\centering
\caption{\label{tab:unnamed-chunk-30}SMA Computations (TSLA.US)}
\centering
\resizebox{\ifdim\width>\linewidth\linewidth\else\width\fi}{!}{
\begin{tabular}[t]{llr>{}r>{}r>{}r>{}r>{}r>{}rr}
\toprule
Symbol & Date & Adj Close & SM5 & SMA8 & SMA13 & SMA20 & SMA50 & SMA200 & Volume\\
\midrule
\cellcolor{gray!10}{TSLA} & \cellcolor{gray!10}{2024-02-09} & \cellcolor{gray!10}{193.6} & \textbf{\cellcolor{gray!10}{187.4}} & \textbf{\cellcolor{gray!10}{187.6}} & \textbf{\cellcolor{gray!10}{189.0}} & \textbf{\cellcolor{gray!10}{197.7}} & \textbf{\cellcolor{gray!10}{225.6}} & \textbf{\cellcolor{gray!10}{232.6}} & \cellcolor{gray!10}{84,476,300}\\
TSLA & 2024-02-12 & 188.1 & \textbf{188.8} & \textbf{187.7} & \textbf{187.5} & \textbf{196.1} & \textbf{224.5} & \textbf{232.8} & 95,498,600\\
\cellcolor{gray!10}{TSLA} & \cellcolor{gray!10}{2024-02-13} & \cellcolor{gray!10}{184.0} & \textbf{\cellcolor{gray!10}{188.6}} & \textbf{\cellcolor{gray!10}{187.1}} & \textbf{\cellcolor{gray!10}{187.6}} & \textbf{\cellcolor{gray!10}{194.3}} & \textbf{\cellcolor{gray!10}{223.3}} & \textbf{\cellcolor{gray!10}{232.9}} & \cellcolor{gray!10}{86,759,500}\\
TSLA & 2024-02-14 & 188.7 & \textbf{188.8} & \textbf{187.2} & \textbf{188.0} & \textbf{193.0} & \textbf{222.3} & \textbf{233.0} & 81,203,000\\
\cellcolor{gray!10}{TSLA} & \cellcolor{gray!10}{2024-02-15} & \cellcolor{gray!10}{200.4} & \textbf{\cellcolor{gray!10}{191.0}} & \textbf{\cellcolor{gray!10}{189.6}} & \textbf{\cellcolor{gray!10}{188.8}} & \textbf{\cellcolor{gray!10}{192.4}} & \textbf{\cellcolor{gray!10}{221.6}} & \textbf{\cellcolor{gray!10}{233.2}} & \cellcolor{gray!10}{120,831,800}\\
TSLA & 2024-02-16 & 199.9 & \textbf{192.3} & \textbf{191.5} & \textbf{189.4} & \textbf{191.8} & \textbf{220.9} & \textbf{233.4} & 111,173,600\\
\cellcolor{gray!10}{TSLA} & \cellcolor{gray!10}{2024-02-20} & \cellcolor{gray!10}{193.8} & \textbf{\cellcolor{gray!10}{193.4}} & \textbf{\cellcolor{gray!10}{192.3}} & \textbf{\cellcolor{gray!10}{189.9}} & \textbf{\cellcolor{gray!10}{191.1}} & \textbf{\cellcolor{gray!10}{220.0}} & \textbf{\cellcolor{gray!10}{233.6}} & \cellcolor{gray!10}{104,545,800}\\
TSLA & 2024-02-21 & 194.8 & \textbf{195.5} & \textbf{192.9} & \textbf{190.4} & \textbf{190.3} & \textbf{219.0} & \textbf{233.7} & 103,844,000\\
\cellcolor{gray!10}{TSLA} & \cellcolor{gray!10}{2024-02-22} & \cellcolor{gray!10}{197.4} & \textbf{\cellcolor{gray!10}{197.3}} & \textbf{\cellcolor{gray!10}{193.4}} & \textbf{\cellcolor{gray!10}{191.1}} & \textbf{\cellcolor{gray!10}{189.8}} & \textbf{\cellcolor{gray!10}{218.1}} & \textbf{\cellcolor{gray!10}{233.9}} & \cellcolor{gray!10}{92,739,500}\\
TSLA & 2024-02-23 & 192.0 & \textbf{195.6} & \textbf{193.9} & \textbf{191.9} & \textbf{190.3} & \textbf{217.1} & \textbf{234.0} & 78,670,300\\
\bottomrule
\multicolumn{10}{l}{\rule{0pt}{1em}\textit{Note: }}\\
\multicolumn{10}{l}{\rule{0pt}{1em}Output 2.3.1.a: Last 10 rows are shown.}\\
\end{tabular}}
\end{table}

The program computed the \textbf{SMA} values for all stocks input in the
model, with the parameter settings of \(n=5,8,13,20,50,200\). As an
illustration, above table shows the computed SMA values (shown above)
and line charts (shown below) for TSLA.US.

翻译

\begin{itemize}
\tightlist
\item
  Trading signal(s) \& strategies- \textbf{SMA}:
\end{itemize}

Normally, when a short-run \textbf{SMA} crosses from below to above a
longer-run SMA is an indication to \textbf{BUY}. When a short-run
\textbf{SMA} crosses from above to below a longer-run \textbf{SMA} is an
indication to \textbf{SELL}.

\begin{itemize}
\item
  \begin{enumerate}
  \def\labelenumi{\arabic{enumi}.}
  \tightlist
  \item
    SMA Trading Signal \#1: \(\text{close price}>\text{SMA}(20)\) (Price
    relative to SMA) {[}\texttt{tsignal1}{]}
  \end{enumerate}
\end{itemize}

When the most recent closing price (the model uses the \emph{adjusted
close} prices) is above its \emph{SMA20} indicates a \textbf{BUY}
signal, and vice versa for a \textbf{SELL} signal. We consider when
\(\text{close price}>\text{SMA}(20)\), the stock is trading at a
strength relative to its recent (over a month) price history.

\begin{itemize}
\item
  \begin{enumerate}
  \def\labelenumi{\arabic{enumi}.}
  \setcounter{enumi}{1}
  \tightlist
  \item
    SMA Trading Signal \#2: \(\text{SMA}(50)>\text{SMA}(200)\)
    (Hierarchical Moving Average Alignment) {[}\texttt{tsignal2}{]}
  \end{enumerate}
\end{itemize}

Above mathematical expression is not completely accurate, where we
believe when \emph{SMA50} crosses from below to above a 200-day Simple
Moving Average (\emph{SMA200}) indicates the stock prices are stable and
established an uptrend trend, i.e.~a \textbf{BUY} signal, and vice
versa. When the \emph{SMA50} crosses from above to below a \emph{SMA200}
indicates a \textbf{SELL} signal. Note, this is one of the most
classical \textbf{SMA} trading strategies, commonly refers to as the
`golden cross' and `death cross'.

\begin{itemize}
\item
  \begin{enumerate}
  \def\labelenumi{\arabic{enumi}.}
  \setcounter{enumi}{2}
  \tightlist
  \item
    SMA Trading Signal \#3: \(\text{SMA}(8)>\text{SMA}(13)\) \&
    \(\text{SMA}(5)>\text{SMA}(8)\) {[}\texttt{tsignal3}{]}
  \end{enumerate}
\end{itemize}

Above mathematical expression is not completely accurate either, where
we believe when \emph{SMA8} crosses from below to above a 13-day Simple
Moving Average (\emph{SMA13}) indicates the stock prices are in an
upward trend for the medium term. If concurrently, the \emph{SMA5}
crosses from below to above a \emph{SMA8}, which indicates the stock
prices are in an upward trend for the short-to-medium term; a
\textbf{BUY} opportunity is signaled.

On the contrast, when the \emph{SMA8} crosses from above to below a
\emph{SMA13}, while the \emph{SMA5} crosses from above to below the
\emph{SMA8}, indicates a \textbf{SELL} signal for the short-to-medium
term.

翻译

To summarize for \textbf{SMA} trading signals:

While comparing \emph{SMA50} and \emph{SMA200} is the more classical and
widely used parameter setting for the simple moving average (SMA)
indicator, namely the golden and death cross. The less conventional,
however modern and effective short-to-medium term trend indicator
settings are the combinations of \(n=5,8,13\).

\begin{itemize}
\item
  SMA- \texttt{tsignal1}:\\
  \textbf{BUY}: \(\text{close price}>\text{SMA}(20)\),\\
  \textbf{HOLD or SELL}: \(\text{close price}\leq \text{SMA}(20)\);
\item
  SMA- \texttt{tsignal2}:\\
  \textbf{BUY}: \(\text{SMA}(50)\) cross from \textbf{below to above}
  \(\text{SMA}(200)\),\\
  \textbf{SELL}: \(\text{SMA}(50)\) cross from \textbf{above to below}
  \(\text{SMA}(200)\);
\item
  SMA- \texttt{tsignal3}:\\
  \textbf{BUY}: \(\text{SMA}(8)\) cross from \textbf{below to above} a
  \(\text{SMA}(13)\) \textbf{and} \(\text{SMA}(5)\) cross from below to
  above a \(\text{SMA}(8)\),\\
  \textbf{SELL}: \(\text{SMA}(8)\) cross from \textbf{above to below} a
  \(\text{SMA}(13)\) \textbf{and} \(\text{SMA}(5)\) cross from below to
  above a \(\text{SMA}(8)\).
\end{itemize}

\textcolor{red}{Individual stock outputs are suppressed, only TSLA.US related results are shown for illustration purposes.个股输出数据及图表已隐藏,报告仅展示特斯拉(TSLA.US)相关结果。}

\emph{\textcolor{red}{Code Hidden 代码已隐藏}}

\begin{itemize}
\tightlist
\item
  Summary- SMA:
\end{itemize}

The program is now constructed to generate \textbf{daily trading
signals} indicates by the \textbf{simple moving average (SMA)}. The
parameter settings and trading rules of the \emph{SMA}-generated
signals, which the model is currently applying are shown above.
\textbf{Daily Trading signals} from the \emph{SMA} indicators, include
\texttt{tsignal1}, \texttt{tsignal2} and \texttt{tsignal3} (include
\texttt{tsignal3a} and \texttt{tsignal3b}) for all stocks are generated,
and the results for TSLA.US is shown below as an illustration.

\(^*:\) Whether the \emph{SMA} indicator, or the trading signals
\texttt{tsignal1}, \texttt{tsignal2} and \texttt{tsignal3} are effective
will be further analyzed by the program, and refer to \(\S 2.4\) below
for more details.

翻译

\begin{longtable}[t]{>{\raggedright\arraybackslash}p{1.9cm}r>{}l>{}l>{}lll}
\caption{\label{tab:unnamed-chunk-39}SMA Trading Signals (TSLA.US)}\\
\toprule
Date & Adj Close & tsignal1 & tsignal2 & tsignal3 & tsignal3a & tsignal3b\\
\midrule
\cellcolor{gray!10}{2024-02-13} & \cellcolor{gray!10}{184.0} & \textbf{\cellcolor{gray!10}{SELL}} & \textbf{\cellcolor{gray!10}{HOLD}} & \textbf{\cellcolor{gray!10}{HOLD}} & \cellcolor{gray!10}{HOLD} & \cellcolor{gray!10}{SELL}\\
2024-02-14 & 188.7 & \textbf{SELL} & \textbf{HOLD} & \textbf{HOLD} & HOLD & HOLD\\
\cellcolor{gray!10}{2024-02-15} & \cellcolor{gray!10}{200.4} & \textbf{\cellcolor{gray!10}{BUY}} & \textbf{\cellcolor{gray!10}{HOLD}} & \textbf{\cellcolor{gray!10}{HOLD}} & \cellcolor{gray!10}{HOLD} & \cellcolor{gray!10}{BUY}\\
2024-02-16 & 199.9 & \textbf{BUY} & \textbf{HOLD} & \textbf{HOLD} & HOLD & HOLD\\
\cellcolor{gray!10}{2024-02-20} & \cellcolor{gray!10}{193.8} & \textbf{\cellcolor{gray!10}{BUY}} & \textbf{\cellcolor{gray!10}{HOLD}} & \textbf{\cellcolor{gray!10}{HOLD}} & \cellcolor{gray!10}{HOLD} & \cellcolor{gray!10}{HOLD}\\
2024-02-21 & 194.8 & \textbf{BUY} & \textbf{HOLD} & \textbf{HOLD} & HOLD & HOLD\\
\cellcolor{gray!10}{2024-02-22} & \cellcolor{gray!10}{197.4} & \textbf{\cellcolor{gray!10}{BUY}} & \textbf{\cellcolor{gray!10}{HOLD}} & \textbf{\cellcolor{gray!10}{HOLD}} & \cellcolor{gray!10}{HOLD} & \cellcolor{gray!10}{HOLD}\\
2024-02-23 & 192.0 & \textbf{BUY} & \textbf{HOLD} & \textbf{HOLD} & HOLD & HOLD\\
\bottomrule
\multicolumn{7}{l}{\rule{0pt}{1em}\textit{Note: }}\\
\multicolumn{7}{l}{\rule{0pt}{1em}Output 2.3.1.b: Last 8 rows are shown.}\\
\multicolumn{7}{l}{\rule{0pt}{1em}\textsuperscript{1} Includes tsignal1, tsignal2, tsignal3, tsignal3a and tsignal3b.}\\
\end{longtable}

\hypertarget{technical-indicator-2-exponential-moving-average-ema}{%
\section{Technical Indicator 2: Exponential Moving Average
(EMA)}\label{technical-indicator-2-exponential-moving-average-ema}}

(Includes \texttt{tsignal4}, \texttt{tsignal5}, and \texttt{tsignal6})

\begin{itemize}
\tightlist
\item
  Intro- \textbf{EMA}:
\end{itemize}

Similar to the Simple Moving Average (SMA) which tend to smooth out the
price variations and the average prices are rolling by dropping the
oldest data point and adding the latest one. An n-day
\textbf{Exponential Moving Average (EMA)}, or the \textbf{n-day EMA}
prices refers to the \textbf{exponential} average of the stock prices
(the model uses the \emph{adjusted close} prices) for the past n
consecutive trading days. Different from the SMA, the
\textbf{Exponential Moving Average (EMA)} use the smoothing factor
\(\beta\) to assign a weight to each data point, with more recent prices
given greater weight because of the exponential decay formula (The
weights can be calculated in various ways, such as linear or
exponential).

Mathematically defined as:

\[\begin{aligned}
\text{EMA}_t(P,n)&=\beta P_{t}+\beta(1-\beta)P_{t-1}+\beta(1-\beta)^{2}P_{t-2}+... \\
&= \beta P_{t}+(1-\beta)\text{EMA}_{t-1}
\end{aligned}\]

where the smoothing coefficient \(\beta\) is usually defined as

\[\beta=\frac{2}{n+1} \in (0;1)\]

The \textbf{EMA} uses the previous value of the EMA
(\(\text{EMA}_{t-1}\)) in its calculation. This means the EMA includes
all the price data within its current value. The smoothing coefficient
ensures that the newest price data has the most impact on the Moving
Average and the oldest prices data has only a minimal impact.

The program uses the formula shown above to compute the
\textbf{Exponential Moving Average (EMA)} values for each stock at
various parameter settings, and creates appropriate related
visualizations. Following, \textbf{EMA} trading signals will be
constructed.

翻译

\textcolor{red}{Individual stock outputs suppressed, only TSLA.US related results shown for illustration purpose.个股输出数据及图表已隐藏,报告仅展示特斯拉(TSLA.US)相关结果。}

\emph{\textcolor{red}{Code Hidden 代码已隐藏}}

\begin{table}[H]
\centering
\caption{\label{tab:unnamed-chunk-44}EMA Computations (TSLA.US)}
\centering
\resizebox{\ifdim\width>\linewidth\linewidth\else\width\fi}{!}{
\begin{tabular}[t]{llr>{}r>{}r>{}r>{}r>{}r>{}rr}
\toprule
Symbol & Date & Adj Close & EM5 & EMA8 & EMA13 & EMA20 & EMA50 & EMA200 & Volume\\
\midrule
\cellcolor{gray!10}{TSLA} & \cellcolor{gray!10}{2024-02-09} & \cellcolor{gray!10}{193.6} & \textbf{\cellcolor{gray!10}{189.6}} & \textbf{\cellcolor{gray!10}{189.9}} & \textbf{\cellcolor{gray!10}{193.1}} & \textbf{\cellcolor{gray!10}{199.2}} & \textbf{\cellcolor{gray!10}{215.8}} & \textbf{\cellcolor{gray!10}{218.5}} & \cellcolor{gray!10}{84,476,300}\\
TSLA & 2024-02-12 & 188.1 & \textbf{189.1} & \textbf{189.5} & \textbf{192.4} & \textbf{198.1} & \textbf{214.7} & \textbf{218.2} & 95,498,600\\
\cellcolor{gray!10}{TSLA} & \cellcolor{gray!10}{2024-02-13} & \cellcolor{gray!10}{184.0} & \textbf{\cellcolor{gray!10}{187.4}} & \textbf{\cellcolor{gray!10}{188.3}} & \textbf{\cellcolor{gray!10}{191.2}} & \textbf{\cellcolor{gray!10}{196.8}} & \textbf{\cellcolor{gray!10}{213.5}} & \textbf{\cellcolor{gray!10}{217.8}} & \cellcolor{gray!10}{86,759,500}\\
TSLA & 2024-02-14 & 188.7 & \textbf{187.8} & \textbf{188.4} & \textbf{190.9} & \textbf{196.0} & \textbf{212.6} & \textbf{217.5} & 81,203,000\\
\cellcolor{gray!10}{TSLA} & \cellcolor{gray!10}{2024-02-15} & \cellcolor{gray!10}{200.4} & \textbf{\cellcolor{gray!10}{192.0}} & \textbf{\cellcolor{gray!10}{191.1}} & \textbf{\cellcolor{gray!10}{192.2}} & \textbf{\cellcolor{gray!10}{196.4}} & \textbf{\cellcolor{gray!10}{212.1}} & \textbf{\cellcolor{gray!10}{217.4}} & \cellcolor{gray!10}{120,831,800}\\
TSLA & 2024-02-16 & 199.9 & \textbf{194.7} & \textbf{193.0} & \textbf{193.3} & \textbf{196.8} & \textbf{211.6} & \textbf{217.2} & 111,173,600\\
\cellcolor{gray!10}{TSLA} & \cellcolor{gray!10}{2024-02-20} & \cellcolor{gray!10}{193.8} & \textbf{\cellcolor{gray!10}{194.4}} & \textbf{\cellcolor{gray!10}{193.2}} & \textbf{\cellcolor{gray!10}{193.4}} & \textbf{\cellcolor{gray!10}{196.5}} & \textbf{\cellcolor{gray!10}{210.9}} & \textbf{\cellcolor{gray!10}{217.0}} & \cellcolor{gray!10}{104,545,800}\\
TSLA & 2024-02-21 & 194.8 & \textbf{194.5} & \textbf{193.5} & \textbf{193.6} & \textbf{196.3} & \textbf{210.3} & \textbf{216.7} & 103,844,000\\
\cellcolor{gray!10}{TSLA} & \cellcolor{gray!10}{2024-02-22} & \cellcolor{gray!10}{197.4} & \textbf{\cellcolor{gray!10}{195.5}} & \textbf{\cellcolor{gray!10}{194.4}} & \textbf{\cellcolor{gray!10}{194.1}} & \textbf{\cellcolor{gray!10}{196.4}} & \textbf{\cellcolor{gray!10}{209.8}} & \textbf{\cellcolor{gray!10}{216.6}} & \cellcolor{gray!10}{92,739,500}\\
TSLA & 2024-02-23 & 192.0 & \textbf{194.3} & \textbf{193.9} & \textbf{193.8} & \textbf{196.0} & \textbf{209.1} & \textbf{216.3} & 78,670,300\\
\bottomrule
\multicolumn{10}{l}{\rule{0pt}{1em}\textit{Note: }}\\
\multicolumn{10}{l}{\rule{0pt}{1em}Output 2.3.2.a: Last 10 rows are shown.}\\
\end{tabular}}
\end{table}

The program computed the \textbf{EMA} values for all stocks input in the
model, with the parameter settings of \(n=5,8,13,20,50,200\). As an
illustration, above table shows the computed EMA values (shown above)
and line charts (shown below) for TSLA.US.

翻译

\begin{itemize}
\tightlist
\item
  Trading signal(s) \& strategies- \textbf{EMA}:
\end{itemize}

Normally, when the \textbf{EMA} rises and prices dip near or just below
the EMA, it signals a buying opportunity. When the EMA falls and prices
rally towards or just above the EMA, it signals a selling opportunity.

Therefore, when a short-run \textbf{EMA} crosses from below to above a
longer-run EMA is an indication to \textbf{BUY}. When a short-run
\textbf{EMA} crosses from above to below a longer-run \textbf{EMA} is an
indication to \textbf{SELL}.

Moving averages are effective in signaling the `support and resistance
areas'. Since the \textbf{EMA} is generally more sensitive to price
movements than the \textbf{SMA}, and is able to identify trends earlier
than an \textbf{SMA} would; the \textbf{EMA} values and resulted trading
signals become more valuable in understanding near-term price movements
and the `support and resistance areas'. A rising EMA tends to support
the price actions, while a falling EMA tends to provide resistance to
price actions. This reinforces the strategy of buying when the price is
near the rising EMA and selling when the price is near the falling EMA.

翻译

\begin{itemize}
\item
  \begin{enumerate}
  \def\labelenumi{\arabic{enumi}.}
  \tightlist
  \item
    EMA Trading Signal \#1: \(\text{close price}>\text{EMA}(20)\) (Price
    relative to EMA) {[}\texttt{tsignal4}{]}
  \end{enumerate}
\end{itemize}

When the most recent closing price (the model uses the \emph{adjusted
close} prices) is above its \emph{EMA20} indicates a \textbf{BUY}
signal, and vice versa for a \textbf{SELL} signal. We consider when
\(\text{close price}>\text{SMA}(20)\), the stock is trading at a
strength relative to its recent (over a month) price history.

\begin{itemize}
\item
  \begin{enumerate}
  \def\labelenumi{\arabic{enumi}.}
  \setcounter{enumi}{1}
  \tightlist
  \item
    EMA Trading Signal \#2: \(\text{EMA}(50)>\text{EMA}(200)\)
    (Hierarchical Moving Average Alignment) {[}\texttt{tsignal5}{]}
  \end{enumerate}
\end{itemize}

Above mathematical expression is not completely accurate, where we
believe when \emph{EMA50} crosses from below to above a 200-day
Exponential Moving Average (\emph{EMA200}) indicates the stock prices
are stable and established an uptrend trend, i.e.~a \textbf{BUY} signal,
and vice versa. When the \emph{EMA50} crosses from above to below a
\emph{EMA200} indicates a \textbf{SELL} signal. Note, this is the
\textbf{EMA} version of the most classical \textbf{SMA} trading
strategies, commonly refers to as the `golden cross' and `death cross'.

\begin{itemize}
\item
  \begin{enumerate}
  \def\labelenumi{\arabic{enumi}.}
  \setcounter{enumi}{2}
  \tightlist
  \item
    EMA Trading Signal \#3: \(\text{EMA}(8)>\text{EMA}(13)\) \&
    \(\text{EMA}(5)>\text{EMA}(8)\) {[}\texttt{tsignal6}{]}
  \end{enumerate}
\end{itemize}

Above mathematical expression is not completely accurate either, where
we believe when \emph{EMA8} crosses from below to above a 13-day
Exponential Moving Average (\emph{EMA13}) indicates the stock prices are
in an upward trend for the medium term. If concurrently, the \emph{EMA5}
crosses from below to above a \emph{EMA8}, which indicates the stock
prices are in an upward trend for the short-to-medium term; a
\textbf{BUY} opportunity is signaled.

On the contrast, when the \emph{EMA8} crosses from above to below an
\emph{EMA13}, while the \emph{EMA5} crosses from above to below the
\emph{EMA8}, indicates a \textbf{SELL} signal for the short-to-medium
term.

Compare to the \textbf{SMA} version, this set of trading signals are
more sensitive, as the \textbf{EMA} bears more weight on the more recent
price variations, and the newest price data has the most impact.

翻译

To summarize for \textbf{EMA} trading signals:

\begin{itemize}
\item
  EMA- \texttt{tsignal4}:\\
  \textbf{BUY}: \(\text{close price}>\text{EMA}(20)\),\\
  \textbf{HOLD or SELL}: \(\text{close price}\leq \text{EMA}(20)\);
\item
  EMA- \texttt{tsignal5}:\\
  \textbf{BUY}: \(\text{EMA}(50)\) cross from \textbf{below to above}
  \(\text{EMA}(200)\),\\
  \textbf{SELL}: \(\text{EMA}(50)\) cross from \textbf{above to below}
  \(\text{EMA}(200)\);
\item
  EMA- \texttt{tsignal6}:\\
  \textbf{BUY}: \(\text{EMA}(8)\) cross from \textbf{below to above} a
  \(\text{EMA}(13)\) \textbf{and} \(\text{EMA}(5)\) cross from below to
  above a \(\text{EMA}(8)\),\\
  \textbf{SELL}: \(\text{EMA}(8)\) cross from \textbf{above to below} a
  \(\text{EMA}(13)\) \textbf{and} \(\text{EMA}(5)\) cross from below to
  above a \(\text{EMA}(8)\)
\end{itemize}

翻译

\textcolor{red}{Individual stock outputs suppressed, only TSLA.US related results shown for illustration purpose.个股输出数据及图表已隐藏,报告仅展示特斯拉(TSLA.US)相关结果。}

\emph{\textcolor{red}{Code Hidden 代码已隐藏}}

~\\
\hspace*{0.333em}

\begin{itemize}
\tightlist
\item
  Summary- \textbf{EMA}:
\end{itemize}

The model uses the same set of trading signal rules or indicator
parameter settings apply to the \textbf{SMA} when interpreting the
\textbf{EMA}. However, the \textbf{EMA} is generally more sensitive to
price movements. On one side, it can signal the trends earlier than an
\textbf{SMA} would. On the other side, the \textbf{EMA} will likely to
exhibit more short-term changes (price fluctuations) than a
corresponding \textbf{SMA}.

All moving averages, including both the \textbf{SMA} and the
\textbf{EMA}; they are not designed to identify the exact bottom and top
(of the price levels). By utilizing the moving averages, the model
intends to observe the general direction of a trend (up,
sideways/congested, down), but the analysts are aware a delay at the
entry and exit points may exist from the moving averages. Most
importantly, the \textbf{EMA} has a shorter delay than the \textbf{SMA}
with the same parameter settings.

翻译

Besides the \textbf{SMA} and the \textbf{EMA}, there are many other
commonly used moving averages such as the \textbf{Weighted Moving
Average (WMA)}, the \textbf{Double Exponential Moving Average (DEMA)},
the \textbf{Triple Exponential Moving Average (TEMA)}, and etc. Most of
them tend to smooth the price movements, so the technical
indicator/trading signal is less sensitive to the short-term
fluctuations. However, per our investment needs in identify intra-day,
short-term and longer-term equity investment opportunities, the model
will therefore adhere to the simple and exponential moving averages.

The program is now constructed to generate \textbf{daily trading
signals} indicates by the \textbf{Exponential Moving Average (EMA)}. The
parameter settings and trading rules of the \emph{EMA}-generated
signals, which the model is currently applying are shown above.
\textbf{Daily Trading signals} from the \emph{EMA} indicators, include
\texttt{tsignal4}, \texttt{tsignal5} and \texttt{tsignal6} (include
\texttt{tsignal6a} and \texttt{tsignal6b}) for all stocks are generated,
and the results for TSLA.US is shown below as an illustration.

\(^*:\) Whether the \emph{EMA} indicator, or the trading signals
\texttt{tsignal4}, \texttt{tsignal5} and \texttt{tsignal6} are effective
will be further analyzed by the program, and refer to \(\S 2.4\) below
for more details.

翻译

\begin{longtable}[t]{>{\raggedright\arraybackslash}p{1.9cm}r>{}l>{}l>{}lll}
\caption{\label{tab:unnamed-chunk-53}EMA Trading Signals (TSLA.US)}\\
\toprule
Date & Adj Close & tsignal4 & tsignal5 & tsignal6 & tsignal6a & tsignal6b\\
\midrule
\cellcolor{gray!10}{2024-02-13} & \cellcolor{gray!10}{184.0} & \textbf{\cellcolor{gray!10}{SELL}} & \textbf{\cellcolor{gray!10}{HOLD}} & \textbf{\cellcolor{gray!10}{HOLD}} & \cellcolor{gray!10}{HOLD} & \cellcolor{gray!10}{HOLD}\\
2024-02-14 & 188.7 & \textbf{SELL} & \textbf{HOLD} & \textbf{HOLD} & HOLD & HOLD\\
\cellcolor{gray!10}{2024-02-15} & \cellcolor{gray!10}{200.4} & \textbf{\cellcolor{gray!10}{BUY}} & \textbf{\cellcolor{gray!10}{HOLD}} & \textbf{\cellcolor{gray!10}{HOLD}} & \cellcolor{gray!10}{BUY} & \cellcolor{gray!10}{HOLD}\\
2024-02-16 & 199.9 & \textbf{BUY} & \textbf{HOLD} & \textbf{HOLD} & HOLD & HOLD\\
\cellcolor{gray!10}{2024-02-20} & \cellcolor{gray!10}{193.8} & \textbf{\cellcolor{gray!10}{SELL}} & \textbf{\cellcolor{gray!10}{HOLD}} & \textbf{\cellcolor{gray!10}{HOLD}} & \cellcolor{gray!10}{HOLD} & \cellcolor{gray!10}{HOLD}\\
2024-02-21 & 194.8 & \textbf{SELL} & \textbf{HOLD} & \textbf{HOLD} & HOLD & HOLD\\
\cellcolor{gray!10}{2024-02-22} & \cellcolor{gray!10}{197.4} & \textbf{\cellcolor{gray!10}{BUY}} & \textbf{\cellcolor{gray!10}{HOLD}} & \textbf{\cellcolor{gray!10}{HOLD}} & \cellcolor{gray!10}{HOLD} & \cellcolor{gray!10}{BUY}\\
2024-02-23 & 192.0 & \textbf{SELL} & \textbf{HOLD} & \textbf{HOLD} & HOLD & HOLD\\
\bottomrule
\multicolumn{7}{l}{\rule{0pt}{1em}\textit{Note: }}\\
\multicolumn{7}{l}{\rule{0pt}{1em}Output 2.3.2.b: Last 8 rows are shown.}\\
\multicolumn{7}{l}{\rule{0pt}{1em}\textsuperscript{1} Includes tsignal4, tsignal5, tsignal6, tsignal6a and tsignal6b.}\\
\end{longtable}

\hypertarget{technical-indicator-3-moving-average-convergence-divergence-macd}{%
\section{Technical Indicator 3: Moving Average Convergence Divergence
(MACD)}\label{technical-indicator-3-moving-average-convergence-divergence-macd}}

\hypertarget{technical-indicator-4-relative-strength-index-rsi}{%
\section{Technical Indicator 4: Relative Strength Index
(RSI)}\label{technical-indicator-4-relative-strength-index-rsi}}

The RSI (Relative Strength Index) is a momentum oscillator that measures
the speed and change of price movements. It ranges from 0 to 100 and is
used to identify overbought or oversold conditions in a market. An RSI
level below 30 generates buy signals while an RSI level above 70
generates sell signals.

Note: The exclusion of machine learning. Technical. Example RSI
Fundamental example ICIR to dig significant momentum factors.

\begin{Shaded}
\begin{Highlighting}[]
\CommentTok{\#}
\end{Highlighting}
\end{Shaded}

\begin{Shaded}
\begin{Highlighting}[]
\CommentTok{\#}
\end{Highlighting}
\end{Shaded}

\begin{Shaded}
\begin{Highlighting}[]
\CommentTok{\#}
\end{Highlighting}
\end{Shaded}

\hypertarget{technical-indicator-5-bollinger-band-bb}{%
\section{Technical Indicator 5: Bollinger Band
(BB)}\label{technical-indicator-5-bollinger-band-bb}}

\texttt{In\ Progress}

\hypertarget{technical-indicator-6-2-day-momentum-2-day-mom}{%
\section{Technical Indicator 6: 2-day Momentum (2-day
Mom)}\label{technical-indicator-6-2-day-momentum-2-day-mom}}

\texttt{In\ Progress}

\hypertarget{technical-indicator-7-price-volume-trend-pvt}{%
\section{Technical Indicator 7: Price Volume Trend
(PVT)}\label{technical-indicator-7-price-volume-trend-pvt}}

\texttt{In\ Progress}

\hypertarget{volume}{%
\subsection{Volume}\label{volume}}

\texttt{In\ Progress}

One of the most common basic indicators traders examine is the trading
volume. Trading volume is an indication for the `activeness' of a
financial instrument. Depending on the financial instruments, trading
volume can be measured either using the number of stocks traded or
number of contracts with changed ownerships.

\hypertarget{price}{%
\subsection{Price}\label{price}}

\texttt{In\ Progress}

To put this in practice, if an increase in volume is observed with a
steady increase in price, the instrument can be viewed as steady and
strong. However, if volume and price are changing in different
directions, a reversal might be happened.

\begin{itemize}
\tightlist
\item
  In terms of trading price, traders often observed the trends based on
  the charts shape and cross in ways that form shapes - often times with
  weird names like `head and shoulder', ' reverse head and sholder',
  `double top', `golden cross', etc. A golden cross indicates a long
  term bull market going forward, whereas the death cross is the exact
  opposite, indicating a potential long term bear market. Both of these
  refer to the confirmation of long term trend by the occurance of the
  overlapping of moving average lines as shown below.
\end{itemize}

\hypertarget{selecting-effective-technical-indicators-ux9009ux53d6ux6709ux6548ux7684ux6280ux672fux6307ux6807}{%
\chapter{Selecting Effective Technical Indicators
选取有效的技术指标}\label{selecting-effective-technical-indicators-ux9009ux53d6ux6709ux6548ux7684ux6280ux672fux6307ux6807}}

\texttt{In\ Progress}

\hypertarget{dynamic-choose-the-effective-technical-indicators}{%
\section{Dynamic: Choose the Effective Technical
Indicators}\label{dynamic-choose-the-effective-technical-indicators}}

\texttt{In\ Progress}

\hypertarget{other-model-construction-steps}{%
\section{Other Model Construction
Steps}\label{other-model-construction-steps}}

\texttt{In\ Progress}

\hypertarget{base-model-performance-ux6a21ux578bux9009ux80a1ux8868ux73b0}{%
\chapter{(Base) Model Performance
模型选股表现}\label{base-model-performance-ux6a21ux578bux9009ux80a1ux8868ux73b0}}

\texttt{In\ Progress}

\hypertarget{model-illustration-stock-selection-using-technical-trade-signals-ux6280ux672fux4fe1ux53f7ux6a21ux578bux9009ux80a1}{%
\section{Model Illustration: Stock Selection Using Technical Trade
Signals
技术信号模型选股}\label{model-illustration-stock-selection-using-technical-trade-signals-ux6280ux672fux4fe1ux53f7ux6a21ux578bux9009ux80a1}}

\texttt{In\ Progress}

\hypertarget{portfolio-simulation-back-test-ux7ec4ux5408ux56deux6d4bux6a21ux62df}{%
\section{Portfolio Simulation (Back-test)
组合回测模拟}\label{portfolio-simulation-back-test-ux7ec4ux5408ux56deux6d4bux6a21ux62df}}

\texttt{In\ Progress}

\hypertarget{model-drawbacks-risks-ux6a21ux578bux7f3aux9677ux548cux98ceux9669}{%
\section{Model Drawbacks \& Risks
模型缺陷和风险}\label{model-drawbacks-risks-ux6a21ux578bux7f3aux9677ux548cux98ceux9669}}

\texttt{In\ Progress}

\hypertarget{full-model-construction-ux5b8cux6574ux6a21ux578bux8bbeux8ba1ux53caux642dux5efa}{%
\chapter{Full Model Construction
完整模型设计及搭建}\label{full-model-construction-ux5b8cux6574ux6a21ux578bux8bbeux8ba1ux53caux642dux5efa}}

TBD

\hypertarget{intro-on-full-model-design-construction-ux5b8cux6574ux91cfux4ef7ux9009ux80a1ux6a21ux578bux642dux5efaux53caux8bbeux8ba1}{%
\section{Intro on Full Model Design \& Construction
完整量价选股模型搭建及设计}\label{intro-on-full-model-design-construction-ux5b8cux6574ux91cfux4ef7ux9009ux80a1ux6a21ux578bux642dux5efaux53caux8bbeux8ba1}}

TBD

\hypertarget{selecting-effective-fundamental-factors-ux9009ux53d6ux6709ux6548ux7684ux57faux672cux9762ux56e0ux5b50}{%
\section{Selecting Effective Fundamental Factors
选取有效的基本面因子}\label{selecting-effective-fundamental-factors-ux9009ux53d6ux6709ux6548ux7684ux57faux672cux9762ux56e0ux5b50}}

TBD

\hypertarget{fundamental-stock-selection-summary-ux57faux672cux9762ux9009ux80a1ux7b56ux7565ux6a21ux578bux7684ux603bux7ed3}{%
\chapter{Fundamental Stock Selection Summary
基本面选股策略模型的总结}\label{fundamental-stock-selection-summary-ux57faux672cux9762ux9009ux80a1ux7b56ux7565ux6a21ux578bux7684ux603bux7ed3}}

TBD

\hypertarget{stock-selection-ux6a21ux578bux9009ux80a1}{%
\section{Stock Selection
模型选股}\label{stock-selection-ux6a21ux578bux9009ux80a1}}

TBD

\hypertarget{portfolio-simulation-back-test-ux7ec4ux5408ux56deux6d4b}{%
\section{Portfolio Simulation (Back-test)
组合回测}\label{portfolio-simulation-back-test-ux7ec4ux5408ux56deux6d4b}}

TBD

\hypertarget{base-model-vs.-full-model-comparison-ux57faux7840ux5b8cux6574ux6a21ux578bux7684ux6bd4ux8f83}{%
\section{Base Model vs.~Full Model Comparison
基础\&完整模型的比较}\label{base-model-vs.-full-model-comparison-ux57faux7840ux5b8cux6574ux6a21ux578bux7684ux6bd4ux8f83}}

TBD

\hypertarget{model-strategy-drawbacks-ux6a21ux578bux548cux7b56ux7565ux8bbeux8ba1ux7f3aux9677}{%
\section{Model \& Strategy Drawbacks
模型和策略设计缺陷}\label{model-strategy-drawbacks-ux6a21ux578bux548cux7b56ux7565ux8bbeux8ba1ux7f3aux9677}}

TBD

\hypertarget{consideration-optimization-using-portfolio-performance-factors-ux5229ux7528ux7ec4ux5408ux56e0ux5b50ux4f18ux5316ux9009ux80a1ux6a21ux578b}{%
\section{Consideration: Optimization Using Portfolio Performance Factors
利用组合因子优化选股模型}\label{consideration-optimization-using-portfolio-performance-factors-ux5229ux7528ux7ec4ux5408ux56e0ux5b50ux4f18ux5316ux9009ux80a1ux6a21ux578b}}

TBD

\hypertarget{stock-selection-strategy-model-summary-ux9009ux80a1ux7b56ux7565ux6a21ux578bux603bux7ed3}{%
\chapter{Stock Selection Strategy \& Model Summary
选股策略模型总结}\label{stock-selection-strategy-model-summary-ux9009ux80a1ux7b56ux7565ux6a21ux578bux603bux7ed3}}

TBD

\hypertarget{asset-allocation-strategy-using-quantitative-models-ux8d44ux4ea7ux914dux7f6eux91cfux5316ux6a21ux578b}{%
\chapter{Asset Allocation Strategy Using Quantitative Models
资产配置量化模型}\label{asset-allocation-strategy-using-quantitative-models-ux8d44ux4ea7ux914dux7f6eux91cfux5316ux6a21ux578b}}

\hypertarget{dynamic-asset-allocation-model---a-quant-tool-to-optimize-portfolio-performance-rm}{%
\section{Dynamic Asset Allocation Model - A Quant tool to Optimize
Portfolio Performance \&
RM}\label{dynamic-asset-allocation-model---a-quant-tool-to-optimize-portfolio-performance-rm}}

TBD

\hypertarget{quantitative-model-for-other-assets-ux91cfux5316ux6a21ux578b}{%
\chapter{Quantitative Model for other Assets
量化模型}\label{quantitative-model-for-other-assets-ux91cfux5316ux6a21ux578b}}

\begin{itemize}
\item
  FX
\item
  Global Gov.~Bonds
\item
  ETFs
\end{itemize}

TBD

\hypertarget{risk-disclosure-ux98ceux9669ux62abux9732}{%
\chapter{Risk Disclosure
风险披露}\label{risk-disclosure-ux98ceux9669ux62abux9732}}

\texttt{In\ Progress}

\hypertarget{risks-equities}{%
\section{Risks: Equities}\label{risks-equities}}

\texttt{In\ Progress}

\hypertarget{risks-other-asset-classes}{%
\section{Risks: Other Asset Classes}\label{risks-other-asset-classes}}

TBD

\hypertarget{portfolio-risks}{%
\section{Portfolio Risks}\label{portfolio-risks}}

\hypertarget{appendix}{%
\chapter{Appendix}\label{appendix}}

\texttt{In\ Progress}

\backmatter
\end{document}
